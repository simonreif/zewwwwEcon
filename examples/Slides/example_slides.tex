%***********************************************************
% ZEW Slides Template
% Simon Reif & Benedikt Stelter, 19 September 2024
%
% Preamble
% This tex code is use to be run together with the
% corresponding Quarto file.
%***********************************************************

% Packages

\documentclass[11pt, aspectratio=169, t]{beamer}
\def\tightlist{}

\RequirePackage{beamerbaserequires}
\usepackage[style=apa]{biblatex}
\addbibresource{references.bib}
\AtBeginBibliography{\footnotesize}

\usepackage[labelfont={color=black,bf}]{caption}
\usepackage{booktabs, longtable, colortbl, array}
\usepackage{threeparttable}
\usepackage{subcaption}

\usepackage[utf8]{inputenc}
\usepackage{graphicx}
\usepackage{tikz}
\usetikzlibrary{decorations.pathreplacing,calc}
\usepackage{color,framed}
\usepackage{xcolor}

\usepackage{amsmath,amssymb}
\usepackage{lmodern}
\usepackage{array}
\usepackage{multirow}
\usepackage{iftex}

\usepackage{unicode-math}
\defaultfontfeatures{Scale=MatchLowercase}
\defaultfontfeatures[\rmfamily]{Ligatures=TeX,Scale=1}

% Use packages from YMAL header:
\usepackage{setspace}
\usepackage{lipsum}



\def\tightlist{}
\definecolor{zewwwwRed}{RGB}{139, 0, 0}
\definecolor{zewwwwGrey}{RGB}{167, 179, 205}

\setbeamercolor{itemize item}{fg=zewwwwGrey}
\setbeamercolor{itemize subitem}{fg=zewwwwGrey}
\setbeamercolor{itemize subsubitem}{fg=zewwwwGrey}

\captionsetup[figure]{labelformat=empty, font=bf}% redefines the caption setup of the figures environment in the beamer class.
\captionsetup[table]{labelformat=empty, font=bf}% redefines the caption setup of the tables environment in the beamer class.

\setbeamertemplate{itemize item}[square]
\setbeamertemplate{itemize subitem}[circle]
\setbeamertemplate{itemize subsubitem}[circle]
\setbeamertemplate{frametitle continuation}{}

\setbeamercolor{bibliography item}{fg=black}
\setbeamercolor{bibliography entry author}{fg=black}
\setbeamercolor{bibliography entry title}{fg=black}
\setbeamercolor{bibliography entry location}{fg=black}
\setbeamercolor{bibliography entry note}{fg=black}
\setbeamertemplate{bibliography item}{}


\setbeamertemplate{blocks}[rounded][shadow=false]
\addtobeamertemplate{block begin}{\pgfsetfillopacity{0.8}}{\pgfsetfillopacity{1}}
\setbeamercolor{structure}{fg=zewwwwRed}
\setbeamercolor*{block title example}{fg=white,
bg= zewwwwRed}
\setbeamercolor*{block body example}{fg= black,
bg= zewwwwGrey!30}


% Insert invers color Section Title Slide
\AtBeginSection{
\addtocounter{framenumber}{0} %Ignore Black Headline Pages for slide count
\usebackgroundtemplate{\includegraphics[width=\paperwidth,height=\paperheight]{zewwwImages/bg.png}}
\begin{frame}[c, noframenumbering, plain]
\begin{minipage}{300pt}
\textcolor{white}{\huge\insertsection}
\end{minipage}
\end{frame}

\usebackgroundtemplate{\includegraphics[width=\paperwidth,height=\paperheight]{zewwwImages/bgmain.pdf}}


\setbeamertemplate{footline}{%
\raisebox{10pt}{\makebox[\paperwidth]{\hfill{\tiny Simon \textcolor{zewwwwRed}{\bfseries{|}} \insertframenumber \makebox[10pt]{} }}}}
}



\makeatletter
\patchcmd\beamer@@tmpl@frametitle{sep=0.8cm}{sep=2cm}{}{}
\makeatother

\beamertemplatenavigationsymbolsempty
\setbeamertemplate{frametitle}{\textcolor{black}{\insertframetitle}}
\addtobeamertemplate{frametitle}{\vskip3ex}{}
\providecommand{\tightlist}{%
	\setlength{\itemsep}{0pt}\setlength{\parskip}{0pt}}

\begin{document}
% Title page
{
\addtocounter{framenumber}{-1} %Ignore Titlepage for slide count
\usebackgroundtemplate{\includegraphics[width=\paperwidth,height=\paperheight]{zewwwImages/bgtitle.png}}
\begin{frame}[c]

\begin{tikzpicture}[remember picture,overlay]
  \node[anchor=south west,inner sep=0pt] at (-0.2,2.5) {
   \includegraphics[height=0.7cm]{zewwwImages/yourlogo.png}
  };
  \node[anchor=south west,inner sep=0pt] at (-0.2,1.8) {
   \includegraphics[height=0.7cm]{zewwwImages/empty.png}
  };
\end{tikzpicture}
\begin{tikzpicture}
\node[overlay, anchor=south west, align=left, text width=0.54\linewidth] at (-0.4,-1.2) {\baselineskip=16pt\textcolor{white}{\textbf{\Large Introducing
the zewwwwEcon Presentation Template}} \par};
\node[overlay, anchor=north west, align=left, text width=9.8cm] at (-0.4,-1.2) { 	\textcolor{white}{Max
Mustermann} \textcolor{zewwwwGrey}{\bfseries{|}}	\textcolor{white}{Institute
X, University ABC} \\
		\textcolor{white}{Erika
Musterfrau} \textcolor{zewwwwGrey}{\bfseries{|}}	\textcolor{white}{Institute
X} \\
	 \vspace{10pt} \textcolor{white}{\textbf{Important
Conference}} \textcolor{zewwwwGrey}{\bfseries{|}} \textbf{\textcolor{white}{23
Sep 2024}}}
;
\end{tikzpicture}
\end{frame}
}

% Main page
\setbeamertemplate{footline}{%
  \raisebox{10pt}{\makebox[\paperwidth]{\hfill{\tiny Simon \textcolor{zewwwwRed}{\bfseries{|}} \insertframenumber \makebox[10pt]{} }}}}

% Define background
\usebackgroundtemplate{\includegraphics[width=\paperwidth,height=\paperheight]{zewwwImages/bgmain.pdf}}



\begin{frame}{Introduction}
\protect\hypertarget{introduction}{}
This Quarto template is supposed to make writing and presenting economic
research easy. Since everything from data to publication is happening in
the same environment, everything is easily reproducible and output can
be modified for the paper and the presentation at the same time.

This example slide covers all aspects necessary for the standard
(economic) researcher:

\begin{itemize}
\tightlist
\item
  Displaying text in different forms
\item
  Handling images
\item
  Graphs that fit the aesthetic of the slides
\item
  Tables 1 to 3 of a standard econ project
\end{itemize}

If you have other useful examples, we can include them in this
presentation so others can benefit as well.
\end{frame}

\hypertarget{the-various-forms-of-displaying-text-in-a-presentation}{%
\section{The various forms of displaying text in a
presentation}\label{the-various-forms-of-displaying-text-in-a-presentation}}

\begin{frame}{Different text inputs}
\protect\hypertarget{different-text-inputs}{}
You can make bullet lists with different levels

\begin{itemize}
\tightlist
\item
  This looks nice and helps separate thoughts
\item
  But you should always have two bullets, otherwise it looks a bit
  weired

  \begin{itemize}
  \tightlist
  \item
    Which to be fair one can argue about

    \begin{itemize}
    \tightlist
    \item
      At least on the third level
    \end{itemize}
  \end{itemize}
\item
  And going back to the first level
\end{itemize}

Sometimes one might need equations. Just use LaTeX for this in the text
to show that \(2^{2} > 3\). You can also have your equation stand out
like this:

\[\hat{\beta} = (X'X)^{-1} X'Y\]
\end{frame}

\begin{frame}{Text and picture side by side using columns}
\protect\hypertarget{text-and-picture-side-by-side-using-columns}{}
There is a very small (0.4\%) column on the left that aligns the first
context column with the headline. These type of slides are in general
not much fun to produce.

\begin{columns}[T]
\begin{column}{0.004\textwidth}
\end{column}

\begin{column}{0.446\textwidth}
\includegraphics[width=\textwidth,height=1.94444in]{pictures/OLSpic.png}
\scriptsize\textbf{Source:} Dall-E drawing a Bauhaus style
representation of the OLS mechanism.
\end{column}

\begin{column}{0.55\textwidth}
\begin{equation}
\frac{\partial S(\beta)}{\partial \beta} = -2X^\top (y - X\beta) = 0
\end{equation}

\begin{equation}
X^\top X \hat{\beta} = X^\top y
\end{equation}

\begin{equation}
\hat{\beta} = (X^\top X)^{-1} X^\top y
\end{equation}

These equations minimize the sum of squared residuals \(S\). While this
seems promising, others argue that this has been done before.
\end{column}
\end{columns}
\end{frame}

\begin{frame}{Highlighting and References}
\protect\hypertarget{highlighting-and-references}{}
Sometimes it seems that not only are people putting books from boxes but
also like boxes around some highlight text. An example would be
something you (can not) find in \textcite{stern2022}:

\begin{exampleblock}{Defining an example}
A lot could be in here. A definition. An equation. Literally there is so much that one would want to get framed with a highlighted headline. 
\end{exampleblock}
\end{frame}

\hypertarget{graphs}{%
\section{Graphs}\label{graphs}}

\begin{frame}{Histogram}
\protect\hypertarget{histogram}{}
\begin{figure}

\caption{\label{fig-hist}Distribution of height (in cm) in random data}

{\centering 

\begin{figure}[H]

{\centering \includegraphics{example_slides_files/figure-beamer/Histogram-1.pdf}

}

\end{figure}

\hypertarget{fig-hist-1}{}
\vspace{-5pt}
\begin{minipage}{0.9\textwidth}
\scriptsize
\singlespacing
\textbf{Notes:} You can use this text to provide further information about the table. \lipsum[66]
\end{minipage}
\vspace{15pt}

}

\end{figure}
\end{frame}

\begin{frame}{Barchart}
\protect\hypertarget{barchart}{}
\begin{figure}

\caption{\label{fig-barchart}Number of Federal States by Country}

{\centering 

\begin{figure}[H]

{\centering \includegraphics{example_slides_files/figure-beamer/Barchart-1.pdf}

}

\end{figure}

\hypertarget{fig-barchart-1}{}
\vspace{-5pt}
\begin{minipage}{0.9\textwidth}
\scriptsize
\singlespacing
\textbf{Notes:} You can use this text to provide further information about the table. \lipsum[66]
\end{minipage}
\vspace{15pt}

}

\end{figure}
\end{frame}

\begin{frame}{Time Series}
\protect\hypertarget{time-series}{}
\begin{figure}

\caption{\label{fig-ts1}Displaying how things evolve over time}

{\centering 

\begin{figure}[H]

{\centering \includegraphics{example_slides_files/figure-beamer/tsplot1-1.pdf}

}

\end{figure}

\hypertarget{fig-ts1-1}{}
\vspace{-5pt}
\begin{minipage}{0.9\textwidth}
\scriptsize
\singlespacing
\textbf{Notes:} You can use this text to provide further information about the table. \lipsum[66]
\end{minipage}
\vspace{15pt}

}

\end{figure}
\end{frame}

\begin{frame}{Scatterplot}
\protect\hypertarget{scatterplot}{}
\begin{figure}

\caption{\label{fig-scatterplot}Two groups have very different values}

{\centering 

\begin{figure}[H]

{\centering \includegraphics{example_slides_files/figure-beamer/scatterplot-1.pdf}

}

\end{figure}

\hypertarget{fig-scatterplot-1}{}
\vspace{-5pt}
\begin{minipage}{0.9\textwidth}
\scriptsize
\singlespacing
\textbf{Notes:} You can use this text to provide further information about the table. \lipsum[66]
\end{minipage}
\vspace{15pt}

}

\end{figure}
\end{frame}

\begin{frame}{Dotplot}
\protect\hypertarget{dotplot}{}
\begin{figure}

\caption{\label{fig-dotplot}Visualizing distributions with few
observations}

{\centering 

\begin{figure}[H]

{\centering \includegraphics{example_slides_files/figure-beamer/dotplot-1.pdf}

}

\end{figure}

\hypertarget{fig-dotplot-1}{}
\vspace{-10pt}
\begin{minipage}{0.9\textwidth}
\scriptsize
\singlespacing
\textbf{Notes:} You can use this text to provide further information about the table. \lipsum[66]
\end{minipage}
\vspace{15pt}

}

\end{figure}
\end{frame}

\begin{frame}{Event Study}
\protect\hypertarget{event-study}{}
\begin{figure}

\caption{\label{fig-eventstudy}Coefficients relative to treatment time}

{\centering 

\begin{figure}[H]

{\centering \includegraphics{example_slides_files/figure-beamer/eventstudy-1.pdf}

}

\end{figure}

\hypertarget{fig-eventstudy-1}{}
\vspace{-10pt}
\begin{minipage}{0.9\textwidth}
\scriptsize
\singlespacing
\textbf{Notes:} You can use this text to provide further information about the table. \lipsum[66]
\end{minipage}
\vspace{15pt}

}

\end{figure}
\end{frame}

\hypertarget{tables}{%
\section{Tables}\label{tables}}

\begin{frame}{Descriptives Table}
\protect\hypertarget{descriptives-table}{}
\hypertarget{tab-descriptives}{}
\begin{longtable}[]{@{}lcc@{}}
\caption{Descriptive statistics by group }\tabularnewline
\toprule\noalign{}
\textbf{Characteristic} & \textbf{0}, N = 265 & \textbf{1}, N = 235 \\
\midrule\noalign{}
\endfirsthead
\toprule\noalign{}
\textbf{Characteristic} & \textbf{0}, N = 265 & \textbf{1}, N = 235 \\
\midrule\noalign{}
\endhead
Survival & 0.16 (0.37) & 0.47 (0.50) \\
Age in years & 49.86 (16.75) & 50.50 (17.93) \\
Female & 0.51 (0.50) & 0.52 (0.50) \\
Severity Score & 0.39 (0.49) & 0.36 (0.48) \\
\bottomrule\noalign{}
\end{longtable}

\vspace{-10pt}
\begin{minipage}{0.9\textwidth}
\scriptsize
\singlespacing
\textbf{Notes:} You can use this text to provide further information about the table. \lipsum[66]
\end{minipage}
\vspace{15pt}
\end{frame}

\begin{frame}{Regression Tables}
\protect\hypertarget{regression-tables}{}
\hypertarget{tab-regression}{}
\begin{longtable}{lcccccc}
\caption{Linear Regression Models}\tabularnewline

\toprule
 & \multicolumn{2}{c}{Full Sample} & \multicolumn{2}{c}{Men} & \multicolumn{2}{c}{Women} \\ 
\cmidrule(lr){2-3} \cmidrule(lr){4-5} \cmidrule(lr){6-7}
  & (I) & (II) & (III) & (IV) & (V) & (VI) \\ 
\midrule\addlinespace[2.5pt]
Treatment & 0.310*** & 0.299*** & 0.337*** & 0.326*** & 0.284*** & 0.267*** \\ 
 & (0.040) & (0.038) & (0.057) & (0.053) & (0.055) & (0.054) \\ 
N & 500 & 500 & 242 & 242 & 258 & 258 \\ 
R² & 0.11 & 0.20 & 0.13 & 0.25 & 0.10 & 0.17 \\ 
\bottomrule
\end{longtable}

\vspace{-10pt}
\begin{minipage}{0.9\textwidth}
\scriptsize
\singlespacing
\textbf{Notes:} You can use this text to provide further information about the table. \lipsum[66]
\end{minipage}
\vspace{15pt}
\end{frame}

\hypertarget{what-we-have-learned}{%
\section{What we have learned}\label{what-we-have-learned}}

\begin{frame}{Last Slide}
\protect\hypertarget{last-slide}{}
This slide will most likely be the one that the audience sees the
longest. Take this into account when designing it. We could for example
point to some things not really working well here until now:

\begin{itemize}
\item
  Display code: It is probably only a small hack but someone needs to do
  it
\item
  Descriptives tables: For some reason there is no easy solution.
\item
  Some equation things do not work: Plotting the LaTeX Symbol conflicts
  with things.
\end{itemize}
\end{frame}
\end{document}
